\documentclass{beamer}
\usepackage{lmodern}
%
% Choose how your presentation looks.
%
% For more themes, color themes and font themes, see:
% http://deic.uab.es/~iblanes/beamer_gallery/index_by_theme.html
%
\mode<presentation>
{
  \usetheme{default}      % or try Darmstadt, Madrid, Warsaw, ...
  \usecolortheme{default} % or try albatross, beaver, crane, ...
  \usefonttheme{default}  % or try serif, structurebold, ...
  \setbeamertemplate{navigation symbols}{}
  \setbeamertemplate{caption}[numbered]
} 

\usepackage[english]{babel}
\usepackage[utf8x]{inputenc}

\title[Avant Internship Review]{Avant Internship Review}
\author{Ben Wang}
\institute{Avant}
\date{\today}

\begin{document}

\begin{frame}
  \titlepage
\end{frame}

% Uncomment these lines for an automatically generated outline.
%\begin{frame}{Outline}
%  \tableofcontents
%\end{frame}


\section{DQF}

\begin{frame}{Data Quality Framework}
Project 1: Handling NaNs
    \begin{block}{Problem}
		\begin{itemize}
		\item Tradelines hourly check reported false positive alters 
		\item  NaNs in the report tables
		\end{itemize}
    \end{block}

    \begin{block}{Root Cause}
		\begin{itemize}
		\item Merged two tables based on a categorical key 
		\item Used NULL values when right table didn't contain a key
        \end{itemize}
    \end{block}

    \begin{block}{Solution}
		\begin{itemize}
		\item Added a NaN-handling step 
		\item replaced all NaNs with 0
		\end{itemize}
    \end{block}
\end{frame}

\begin{frame}{Data Quality Framework}
Project 2: Created Data Quality Checks for Credit Decision and Product Decision
    \begin{block}{My work}
		\begin{itemize}
		\item Wrote SQL queries to select inputs and calculate summary
        statistics 
		\item Merged the queries into DQF to create garden jobs 
		\end{itemize}
    \end{block}

    \begin{block}{Difficulty}
		\begin{itemize}
		\item All inputs nested in to one column as a String 
        \end{itemize}
    \end{block}

    \begin{block}{Solution}
		\begin{itemize}
		\item Used $plit\_part$ and $substring$ to phrase inputs into individual
        into different variables 
		\end{itemize}
    \end{block}
\end{frame}
\section{Direct Mail}
\begin{frame}{Direct Mail}
Project 3: Fixed garden jobs that create suppression files 
    \begin{block}{Problem}
		\begin{itemize}
		\item Decline garden job could not finish  
		\end{itemize}
    \end{block}

    \begin{block}{Root Cause}
		\begin{itemize}
		\item The job's query was interrupted by followers 
        \end{itemize}
    \end{block}

    \begin{block}{Solution}
		\begin{itemize}
		\item Read the origina lquery and document the locical structure in
        Conflunece 
		\item Translate query to Hive sql
        \item Validate results of Psql and Hive query
		\end{itemize}
    \end{block}
\end{frame}
\section{Introduction}

\begin{frame}{}

\begin{itemize}
  \item Your introduction goes here!
  \item Use \texttt{itemize} to organize your main points.
\end{itemize}

\vskip 1cm

\begin{block}{Examples}
Some examples of commonly used commands and features are included, to help you get started.
\end{block}

\end{frame}

\section{Some \LaTeX{} Examples}

\subsection{Tables and Figures}

\begin{frame}{Tables and Figures}

\begin{itemize}
\item Use \texttt{tabular} for basic tables --- see Table~\ref{tab:widgets}, for example.
\item You can upload a figure (JPEG, PNG or PDF) using the files menu. 
\item To include it in your document, use the \texttt{includegraphics} command (see the comment below in the source code).
\end{itemize}

% Commands to include a figure:
%\begin{figure}
%\includegraphics[width=\textwidth]{your-figure's-file-name}
%\caption{\label{fig:your-figure}Caption goes here.}
%\end{figure}

\begin{table}
\centering
\begin{tabular}{l|r}
Item & Quantity \\\hline
Widgets & 42 \\
Gadgets & 13
\end{tabular}
\caption{\label{tab:widgets}An example table.}
\end{table}

\end{frame}

\subsection{Mathematics}

\begin{frame}{Readable Mathematics}

Let $X_1, X_2, \ldots, X_n$ be a sequence of independent and identically distributed random variables with $\text{E}[X_i] = \mu$ and $\text{Var}[X_i] = \sigma^2 < \infty$, and let
$$S_n = \frac{X_1 + X_2 + \cdots + X_n}{n}
      = \frac{1}{n}\sum_{i}^{n} X_i$$
denote their mean. Then as $n$ approaches infinity, the random variables $\sqrt{n}(S_n - \mu)$ converge in distribution to a normal $\mathcal{N}(0, \sigma^2)$.

\end{frame}

\end{document}

\end{document}
