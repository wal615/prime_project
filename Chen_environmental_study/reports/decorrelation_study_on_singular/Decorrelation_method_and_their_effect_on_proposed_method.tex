\documentclass[]{article}
\usepackage{lmodern}
\usepackage{amssymb,amsmath}
\usepackage{ifxetex,ifluatex}
\usepackage{fixltx2e} % provides \textsubscript
\ifnum 0\ifxetex 1\fi\ifluatex 1\fi=0 % if pdftex
  \usepackage[T1]{fontenc}
  \usepackage[utf8]{inputenc}
\else % if luatex or xelatex
  \ifxetex
    \usepackage{mathspec}
  \else
    \usepackage{fontspec}
  \fi
  \defaultfontfeatures{Ligatures=TeX,Scale=MatchLowercase}
\fi
% use upquote if available, for straight quotes in verbatim environments
\IfFileExists{upquote.sty}{\usepackage{upquote}}{}
% use microtype if available
\IfFileExists{microtype.sty}{%
\usepackage{microtype}
\UseMicrotypeSet[protrusion]{basicmath} % disable protrusion for tt fonts
}{}
\usepackage[margin=1in]{geometry}
\usepackage{hyperref}
\hypersetup{unicode=true,
            pdftitle={Decorrelation methods and their effects on proposed method},
            pdfauthor={Xuelong Wang},
            pdfborder={0 0 0},
            breaklinks=true}
\urlstyle{same}  % don't use monospace font for urls
\usepackage{color}
\usepackage{fancyvrb}
\newcommand{\VerbBar}{|}
\newcommand{\VERB}{\Verb[commandchars=\\\{\}]}
\DefineVerbatimEnvironment{Highlighting}{Verbatim}{commandchars=\\\{\}}
% Add ',fontsize=\small' for more characters per line
\usepackage{framed}
\definecolor{shadecolor}{RGB}{248,248,248}
\newenvironment{Shaded}{\begin{snugshade}}{\end{snugshade}}
\newcommand{\KeywordTok}[1]{\textcolor[rgb]{0.13,0.29,0.53}{\textbf{#1}}}
\newcommand{\DataTypeTok}[1]{\textcolor[rgb]{0.13,0.29,0.53}{#1}}
\newcommand{\DecValTok}[1]{\textcolor[rgb]{0.00,0.00,0.81}{#1}}
\newcommand{\BaseNTok}[1]{\textcolor[rgb]{0.00,0.00,0.81}{#1}}
\newcommand{\FloatTok}[1]{\textcolor[rgb]{0.00,0.00,0.81}{#1}}
\newcommand{\ConstantTok}[1]{\textcolor[rgb]{0.00,0.00,0.00}{#1}}
\newcommand{\CharTok}[1]{\textcolor[rgb]{0.31,0.60,0.02}{#1}}
\newcommand{\SpecialCharTok}[1]{\textcolor[rgb]{0.00,0.00,0.00}{#1}}
\newcommand{\StringTok}[1]{\textcolor[rgb]{0.31,0.60,0.02}{#1}}
\newcommand{\VerbatimStringTok}[1]{\textcolor[rgb]{0.31,0.60,0.02}{#1}}
\newcommand{\SpecialStringTok}[1]{\textcolor[rgb]{0.31,0.60,0.02}{#1}}
\newcommand{\ImportTok}[1]{#1}
\newcommand{\CommentTok}[1]{\textcolor[rgb]{0.56,0.35,0.01}{\textit{#1}}}
\newcommand{\DocumentationTok}[1]{\textcolor[rgb]{0.56,0.35,0.01}{\textbf{\textit{#1}}}}
\newcommand{\AnnotationTok}[1]{\textcolor[rgb]{0.56,0.35,0.01}{\textbf{\textit{#1}}}}
\newcommand{\CommentVarTok}[1]{\textcolor[rgb]{0.56,0.35,0.01}{\textbf{\textit{#1}}}}
\newcommand{\OtherTok}[1]{\textcolor[rgb]{0.56,0.35,0.01}{#1}}
\newcommand{\FunctionTok}[1]{\textcolor[rgb]{0.00,0.00,0.00}{#1}}
\newcommand{\VariableTok}[1]{\textcolor[rgb]{0.00,0.00,0.00}{#1}}
\newcommand{\ControlFlowTok}[1]{\textcolor[rgb]{0.13,0.29,0.53}{\textbf{#1}}}
\newcommand{\OperatorTok}[1]{\textcolor[rgb]{0.81,0.36,0.00}{\textbf{#1}}}
\newcommand{\BuiltInTok}[1]{#1}
\newcommand{\ExtensionTok}[1]{#1}
\newcommand{\PreprocessorTok}[1]{\textcolor[rgb]{0.56,0.35,0.01}{\textit{#1}}}
\newcommand{\AttributeTok}[1]{\textcolor[rgb]{0.77,0.63,0.00}{#1}}
\newcommand{\RegionMarkerTok}[1]{#1}
\newcommand{\InformationTok}[1]{\textcolor[rgb]{0.56,0.35,0.01}{\textbf{\textit{#1}}}}
\newcommand{\WarningTok}[1]{\textcolor[rgb]{0.56,0.35,0.01}{\textbf{\textit{#1}}}}
\newcommand{\AlertTok}[1]{\textcolor[rgb]{0.94,0.16,0.16}{#1}}
\newcommand{\ErrorTok}[1]{\textcolor[rgb]{0.64,0.00,0.00}{\textbf{#1}}}
\newcommand{\NormalTok}[1]{#1}
\usepackage{graphicx,grffile}
\makeatletter
\def\maxwidth{\ifdim\Gin@nat@width>\linewidth\linewidth\else\Gin@nat@width\fi}
\def\maxheight{\ifdim\Gin@nat@height>\textheight\textheight\else\Gin@nat@height\fi}
\makeatother
% Scale images if necessary, so that they will not overflow the page
% margins by default, and it is still possible to overwrite the defaults
% using explicit options in \includegraphics[width, height, ...]{}
\setkeys{Gin}{width=\maxwidth,height=\maxheight,keepaspectratio}
\IfFileExists{parskip.sty}{%
\usepackage{parskip}
}{% else
\setlength{\parindent}{0pt}
\setlength{\parskip}{6pt plus 2pt minus 1pt}
}
\setlength{\emergencystretch}{3em}  % prevent overfull lines
\providecommand{\tightlist}{%
  \setlength{\itemsep}{0pt}\setlength{\parskip}{0pt}}
\setcounter{secnumdepth}{5}
% Redefines (sub)paragraphs to behave more like sections
\ifx\paragraph\undefined\else
\let\oldparagraph\paragraph
\renewcommand{\paragraph}[1]{\oldparagraph{#1}\mbox{}}
\fi
\ifx\subparagraph\undefined\else
\let\oldsubparagraph\subparagraph
\renewcommand{\subparagraph}[1]{\oldsubparagraph{#1}\mbox{}}
\fi

%%% Use protect on footnotes to avoid problems with footnotes in titles
\let\rmarkdownfootnote\footnote%
\def\footnote{\protect\rmarkdownfootnote}

%%% Change title format to be more compact
\usepackage{titling}

% Create subtitle command for use in maketitle
\newcommand{\subtitle}[1]{
  \posttitle{
    \begin{center}\large#1\end{center}
    }
}

\setlength{\droptitle}{-2em}

  \title{Decorrelation methods and their effects on proposed method}
    \pretitle{\vspace{\droptitle}\centering\huge}
  \posttitle{\par}
    \author{Xuelong Wang}
    \preauthor{\centering\large\emph}
  \postauthor{\par}
      \predate{\centering\large\emph}
  \postdate{\par}
    \date{2018-10-04}

\usepackage{float,amsmath, bbm, siunitx, bm}
\floatplacement{figure}{H}
\newcommand{\indep}{\rotatebox[origin=c]{90}{$\models$}}

\begin{document}
\maketitle

{
\setcounter{tocdepth}{2}
\tableofcontents
}
\section{Motivation}\label{motivation}

Based on the previous simulation result, we found that the decorrelation
step has a big influence on the final performance of the proposed
method. More sepcifically, when the \(n<p\) is happening then we known
that the sample covariance matrix \(\Sigma_{X}\) is not full rank.
Therefore, \(\Sigma^{-1}_X\), the inverse of \(\Sigma_{X}\), doesn't
exist. So we could calculate the general inverse of the covariance
matrix \(\Sigma_{X}\). In such situation, I just adapted one of commonly
used g-inverse -- the moore penrose inverse \(\Sigma^{+}_X\) during the
decorrelation procedure. But the result is not very well compared with
the original method. Thus, the following is trying to discuss the reason
of why this is not working.

\section{SVD decorrelation procedure}\label{svd-decorrelation-procedure}

\[
  Var(X) = \Sigma_X = U\Lambda V^T,
\]

\begin{itemize}
\tightlist
\item
  \(X\) is the random vector with dim as \(p \times 1\),\\
\item
  \(\Sigma_X\) is \(p \times p\) symmtery matrix,\\
\item
  \(U = V\) are orthogonal matrix and each column is the eigenvector\\
\item
  \(\Lambda\) is a diagonal matrix with each diagonal element as the
  eigenvalue.
\end{itemize}

\subsection{\texorpdfstring{Assume the \(\Sigma_X\) is full
rank}{Assume the \textbackslash{}Sigma\_X is full rank}}\label{assume-the-sigma_x-is-full-rank}

To decorreate the X, we could just take the reciprocal of each square
root of eigenvalue as following.

\[
  \Sigma^{-1/2}_X = U\Lambda^{-1/2}U^T,
\] where
\(\Lambda^{-1/2} = \begin{bmatrix}  e_1^{-1/2} & \dots & 0 \\  \vdots & \ddots & \vdots \\  0 & \dots & e_p^{-1/2}  \end{bmatrix}\)

So that after transformation the \(\Sigma^{-1/2}_X X\) has identity
covariance matrix as following,

\[
  Var(\Sigma^{-1/2}_X X) = \Sigma^{-1/2}_X \Sigma_X\Sigma^{-1/2}_X = U\Lambda^{-1/2}U^T U\Lambda^{-1}U^T U\Lambda^{-1/2}U^T = I_p.
\]

\subsection{\texorpdfstring{Assume the \(\Sigma_X\) is not full
rank}{Assume the \textbackslash{}Sigma\_X is not full rank}}\label{assume-the-sigma_x-is-not-full-rank}

\[
  Var(X) = \Sigma_X = U\Lambda V^T =
                        \begin{bmatrix}
                         U_1 & U_2\\
                        \end{bmatrix}
                        \begin{bmatrix}
                        \Lambda_1 & 0\\
                        0 & 0
                        \end{bmatrix}
                        \begin{bmatrix}
                        U_1^T \\
                        U_2^T
                        \end{bmatrix} = U_1\Lambda_1U_1^T,
\] - \(U_1\) is a \(p \times r\) matrix with r \textless{} p and in most
of case r = n the sample size.

Then after applying the same procedure we get following,

\[
  \Sigma^{-1/2}_X = U_1\Lambda_1^{-1/2}U_1^T,
\] Note that in this case, I'm using Moore Penrose inverse.

After transformation the X we have, \[
  Var(\Sigma^{-1/2}_X X) = \Sigma^{-1/2}_X \Sigma_X\Sigma^{-1/2}_X = U_1\Lambda^{-1/2}_1U^T_1 U_1\Lambda^{-1}_1U^T_1 U_1\Lambda^{-1/2}_1U^T_1 = U_1U_1^T,  
\] Note that by the property of the U we have \[
  U_1U_1^T + U_2U_2^T = I_p \\
  (U_1U_1^T)^T U_1U_1^T = U_1U_1^T,
\] Besides, \(U_1U_1^T\) and \(U_2U_2^T\) are indempotent and
\(rank(U_2U_2^T) + rank(U_1U_1^T) = p\).

So if the X is not full rank we cannot decorrelation the covariance
matrix to an identity matrix.

\subsection{Simulation stduy}\label{simulation-stduy}

\subsubsection{Simulation 1}\label{simulation-1}

\begin{Shaded}
\begin{Highlighting}[]
\CommentTok{# How the singular sample covariance affect the SVD decorrelation result}
\KeywordTok{set.seed}\NormalTok{(}\DecValTok{123}\NormalTok{)}
\NormalTok{p <-}\StringTok{ }\DecValTok{200}
\NormalTok{n <-}\StringTok{ }\DecValTok{200}
\NormalTok{Sig <-}\StringTok{ }\KeywordTok{matrix}\NormalTok{(}\KeywordTok{rep}\NormalTok{(}\FloatTok{0.5}\NormalTok{, }\DecValTok{200} \OperatorTok{*}\StringTok{ }\DecValTok{200}\NormalTok{), }\DataTypeTok{ncol =} \DecValTok{200}\NormalTok{)}
\KeywordTok{diag}\NormalTok{(Sig) <-}\StringTok{ }\DecValTok{1}
\NormalTok{x_total <-}\StringTok{ }\KeywordTok{mvrnorm}\NormalTok{(n, }\KeywordTok{numeric}\NormalTok{(p), }\DataTypeTok{Sigma =}\NormalTok{ Sig)}

\NormalTok{x_}\DecValTok{100}\NormalTok{ <-}\StringTok{ }\NormalTok{x_total[}\DecValTok{1}\OperatorTok{:}\DecValTok{100}\NormalTok{, ]}
\NormalTok{Est_sqrt_ins_cov_}\DecValTok{100}\NormalTok{ <-}\StringTok{ }\KeywordTok{invsqrt}\NormalTok{(}\KeywordTok{cov}\NormalTok{(x_}\DecValTok{100}\NormalTok{))}
\KeywordTok{cor}\NormalTok{(x_}\DecValTok{100} \OperatorTok\StringTok{ }\NormalTok{Est_sqrt_ins_cov_}\DecValTok{100}\NormalTok{)[}\DecValTok{1}\OperatorTok{:}\DecValTok{5}\NormalTok{, }\DecValTok{1}\OperatorTok{:}\DecValTok{5}\NormalTok{] }\OperatorTok\StringTok{ }\KeywordTok{round}\NormalTok{(., }\DecValTok{4}\NormalTok{)}
\OtherTok{FALSE}\NormalTok{         [,}\DecValTok{1}\NormalTok{]    [,}\DecValTok{2}\NormalTok{]    [,}\DecValTok{3}\NormalTok{]   [,}\DecValTok{4}\NormalTok{]    [,}\DecValTok{5}\NormalTok{]}
\OtherTok{FALSE}\NormalTok{ [}\DecValTok{1}\NormalTok{,]  }\FloatTok{1.0000} \OperatorTok{-}\FloatTok{0.0480} \OperatorTok{-}\FloatTok{0.0600} \FloatTok{0.0396} \OperatorTok{-}\FloatTok{0.0232}
\OtherTok{FALSE}\NormalTok{ [}\DecValTok{2}\NormalTok{,] }\OperatorTok{-}\FloatTok{0.0480}  \FloatTok{1.0000} \OperatorTok{-}\FloatTok{0.0339} \FloatTok{0.1562} \OperatorTok{-}\FloatTok{0.0701}
\OtherTok{FALSE}\NormalTok{ [}\DecValTok{3}\NormalTok{,] }\OperatorTok{-}\FloatTok{0.0600} \OperatorTok{-}\FloatTok{0.0339}  \FloatTok{1.0000} \FloatTok{0.0389} \OperatorTok{-}\FloatTok{0.0570}
\OtherTok{FALSE}\NormalTok{ [}\DecValTok{4}\NormalTok{,]  }\FloatTok{0.0396}  \FloatTok{0.1562}  \FloatTok{0.0389} \FloatTok{1.0000}  \FloatTok{0.0734}
\OtherTok{FALSE}\NormalTok{ [}\DecValTok{5}\NormalTok{,] }\OperatorTok{-}\FloatTok{0.0232} \OperatorTok{-}\FloatTok{0.0701} \OperatorTok{-}\FloatTok{0.0570} \FloatTok{0.0734}  \FloatTok{1.0000}
\KeywordTok{cor}\NormalTok{(x_}\DecValTok{100} \OperatorTok\StringTok{ }\NormalTok{Est_sqrt_ins_cov_}\DecValTok{100}\NormalTok{) }\OperatorTok\StringTok{ }\KeywordTok{abs}\NormalTok{(.) }\OperatorTok\StringTok{ }\KeywordTok{sum}\NormalTok{(.)}
\OtherTok{FALSE}\NormalTok{ [}\DecValTok{1}\NormalTok{] }\FloatTok{2480.243}
\KeywordTok{cor}\NormalTok{(x_}\DecValTok{100} \OperatorTok\StringTok{ }\NormalTok{Est_sqrt_ins_cov_}\DecValTok{100}\NormalTok{) }\OperatorTok\StringTok{ }\KeywordTok{diag}\NormalTok{(.) }\OperatorTok\StringTok{ }\KeywordTok{sum}\NormalTok{(.)}
\OtherTok{FALSE}\NormalTok{ [}\DecValTok{1}\NormalTok{] }\DecValTok{200}
\KeywordTok{cor}\NormalTok{(x_}\DecValTok{100} \OperatorTok\StringTok{ }\NormalTok{Est_sqrt_ins_cov_}\DecValTok{100}\NormalTok{)[}\KeywordTok{cor}\NormalTok{(x_}\DecValTok{100} \OperatorTok\StringTok{ }\NormalTok{Est_sqrt_ins_cov_}\DecValTok{100}\NormalTok{) }\OperatorTok\StringTok{ }
\StringTok{    }\KeywordTok{lower.tri}\NormalTok{(., }\DataTypeTok{diag =} \OtherTok{FALSE}\NormalTok{)] }\OperatorTok\StringTok{ }\KeywordTok{max}\NormalTok{()}
\OtherTok{FALSE}\NormalTok{ [}\DecValTok{1}\NormalTok{] }\FloatTok{0.2839148}

\NormalTok{x_}\DecValTok{200}\NormalTok{ <-}\StringTok{ }\NormalTok{x_total}
\NormalTok{Est_sqrt_ins_cov_}\DecValTok{200}\NormalTok{ <-}\StringTok{ }\KeywordTok{invsqrt}\NormalTok{(}\KeywordTok{cov}\NormalTok{(x_}\DecValTok{200}\NormalTok{))}
\KeywordTok{cor}\NormalTok{(x_}\DecValTok{200} \OperatorTok\StringTok{ }\NormalTok{Est_sqrt_ins_cov_}\DecValTok{200}\NormalTok{)[}\DecValTok{1}\OperatorTok{:}\DecValTok{5}\NormalTok{, }\DecValTok{1}\OperatorTok{:}\DecValTok{5}\NormalTok{] }\OperatorTok\StringTok{ }\KeywordTok{round}\NormalTok{(., }\DecValTok{4}\NormalTok{)}
\OtherTok{FALSE}\NormalTok{         [,}\DecValTok{1}\NormalTok{]    [,}\DecValTok{2}\NormalTok{]    [,}\DecValTok{3}\NormalTok{]   [,}\DecValTok{4}\NormalTok{]    [,}\DecValTok{5}\NormalTok{]}
\OtherTok{FALSE}\NormalTok{ [}\DecValTok{1}\NormalTok{,]  }\FloatTok{1.0000}  \FloatTok{0.0032}  \FloatTok{0.0024} \OperatorTok{-}\FloatTok{3e-04}  \FloatTok{0.0034}
\OtherTok{FALSE}\NormalTok{ [}\DecValTok{2}\NormalTok{,]  }\FloatTok{0.0032}  \FloatTok{1.0000} \OperatorTok{-}\FloatTok{0.0049}  \FloatTok{6e-04} \OperatorTok{-}\FloatTok{0.0069}
\OtherTok{FALSE}\NormalTok{ [}\DecValTok{3}\NormalTok{,]  }\FloatTok{0.0024} \OperatorTok{-}\FloatTok{0.0049}  \FloatTok{1.0000}  \FloatTok{4e-04} \OperatorTok{-}\FloatTok{0.0052}
\OtherTok{FALSE}\NormalTok{ [}\DecValTok{4}\NormalTok{,] }\OperatorTok{-}\FloatTok{0.0003}  \FloatTok{0.0006}  \FloatTok{0.0004}  \FloatTok{1e+00}  \FloatTok{0.0006}
\OtherTok{FALSE}\NormalTok{ [}\DecValTok{5}\NormalTok{,]  }\FloatTok{0.0034} \OperatorTok{-}\FloatTok{0.0069} \OperatorTok{-}\FloatTok{0.0052}  \FloatTok{6e-04}  \FloatTok{1.0000}
\KeywordTok{cor}\NormalTok{(x_}\DecValTok{200} \OperatorTok\StringTok{ }\NormalTok{Est_sqrt_ins_cov_}\DecValTok{200}\NormalTok{) }\OperatorTok\StringTok{ }\KeywordTok{abs}\NormalTok{(.) }\OperatorTok\StringTok{ }\KeywordTok{sum}\NormalTok{(.)}
\OtherTok{FALSE}\NormalTok{ [}\DecValTok{1}\NormalTok{] }\FloatTok{327.6482}
\KeywordTok{cor}\NormalTok{(x_}\DecValTok{200} \OperatorTok\StringTok{ }\NormalTok{Est_sqrt_ins_cov_}\DecValTok{200}\NormalTok{) }\OperatorTok\StringTok{ }\KeywordTok{diag}\NormalTok{(.) }\OperatorTok\StringTok{ }\KeywordTok{sum}\NormalTok{(.)}
\OtherTok{FALSE}\NormalTok{ [}\DecValTok{1}\NormalTok{] }\DecValTok{200}
\KeywordTok{cor}\NormalTok{(x_}\DecValTok{200} \OperatorTok\StringTok{ }\NormalTok{Est_sqrt_ins_cov_}\DecValTok{200}\NormalTok{)[}\KeywordTok{cor}\NormalTok{(x_}\DecValTok{200} \OperatorTok\StringTok{ }\NormalTok{Est_sqrt_ins_cov_}\DecValTok{200}\NormalTok{) }\OperatorTok\StringTok{ }
\StringTok{    }\KeywordTok{lower.tri}\NormalTok{(., }\DataTypeTok{diag =} \OtherTok{FALSE}\NormalTok{)] }\OperatorTok\StringTok{ }\KeywordTok{max}\NormalTok{()}
\OtherTok{FALSE}\NormalTok{ [}\DecValTok{1}\NormalTok{] }\FloatTok{0.03669759}

\CommentTok{# if we use the inverse information of x_200}
\KeywordTok{cor}\NormalTok{(x_}\DecValTok{100} \OperatorTok\StringTok{ }\NormalTok{Est_sqrt_ins_cov_}\DecValTok{200}\NormalTok{)[}\DecValTok{1}\OperatorTok{:}\DecValTok{5}\NormalTok{, }\DecValTok{1}\OperatorTok{:}\DecValTok{5}\NormalTok{] }\OperatorTok\StringTok{ }\KeywordTok{round}\NormalTok{(., }\DecValTok{4}\NormalTok{)}
\OtherTok{FALSE}\NormalTok{         [,}\DecValTok{1}\NormalTok{]    [,}\DecValTok{2}\NormalTok{]    [,}\DecValTok{3}\NormalTok{]    [,}\DecValTok{4}\NormalTok{]    [,}\DecValTok{5}\NormalTok{]}
\OtherTok{FALSE}\NormalTok{ [}\DecValTok{1}\NormalTok{,]  }\FloatTok{1.0000} \OperatorTok{-}\FloatTok{0.0065} \OperatorTok{-}\FloatTok{0.1252} \OperatorTok{-}\FloatTok{0.0091} \OperatorTok{-}\FloatTok{0.0865}
\OtherTok{FALSE}\NormalTok{ [}\DecValTok{2}\NormalTok{,] }\OperatorTok{-}\FloatTok{0.0065}  \FloatTok{1.0000} \OperatorTok{-}\FloatTok{0.0076}  \FloatTok{0.2172} \OperatorTok{-}\FloatTok{0.0999}
\OtherTok{FALSE}\NormalTok{ [}\DecValTok{3}\NormalTok{,] }\OperatorTok{-}\FloatTok{0.1252} \OperatorTok{-}\FloatTok{0.0076}  \FloatTok{1.0000}  \FloatTok{0.0741} \OperatorTok{-}\FloatTok{0.0986}
\OtherTok{FALSE}\NormalTok{ [}\DecValTok{4}\NormalTok{,] }\OperatorTok{-}\FloatTok{0.0091}  \FloatTok{0.2172}  \FloatTok{0.0741}  \FloatTok{1.0000}  \FloatTok{0.0489}
\OtherTok{FALSE}\NormalTok{ [}\DecValTok{5}\NormalTok{,] }\OperatorTok{-}\FloatTok{0.0865} \OperatorTok{-}\FloatTok{0.0999} \OperatorTok{-}\FloatTok{0.0986}  \FloatTok{0.0489}  \FloatTok{1.0000}
\KeywordTok{cor}\NormalTok{(x_}\DecValTok{100} \OperatorTok\StringTok{ }\NormalTok{Est_sqrt_ins_cov_}\DecValTok{200}\NormalTok{) }\OperatorTok\StringTok{ }\KeywordTok{abs}\NormalTok{(.) }\OperatorTok\StringTok{ }\KeywordTok{sum}\NormalTok{(.)}
\OtherTok{FALSE}\NormalTok{ [}\DecValTok{1}\NormalTok{] }\FloatTok{2479.92}
\KeywordTok{cor}\NormalTok{(x_}\DecValTok{100} \OperatorTok\StringTok{ }\NormalTok{Est_sqrt_ins_cov_}\DecValTok{200}\NormalTok{) }\OperatorTok\StringTok{ }\KeywordTok{diag}\NormalTok{(.) }\OperatorTok\StringTok{ }\KeywordTok{sum}\NormalTok{(.)}
\OtherTok{FALSE}\NormalTok{ [}\DecValTok{1}\NormalTok{] }\DecValTok{200}
\KeywordTok{cor}\NormalTok{(x_}\DecValTok{100} \OperatorTok\StringTok{ }\NormalTok{Est_sqrt_ins_cov_}\DecValTok{200}\NormalTok{)[}\KeywordTok{cor}\NormalTok{(x_}\DecValTok{100} \OperatorTok\StringTok{ }\NormalTok{Est_sqrt_ins_cov_}\DecValTok{200}\NormalTok{) }\OperatorTok\StringTok{ }
\StringTok{    }\KeywordTok{lower.tri}\NormalTok{(., }\DataTypeTok{diag =} \OtherTok{FALSE}\NormalTok{)] }\OperatorTok\StringTok{ }\KeywordTok{max}\NormalTok{()}
\OtherTok{FALSE}\NormalTok{ [}\DecValTok{1}\NormalTok{] }\FloatTok{0.2662189}
\end{Highlighting}
\end{Shaded}

\begin{itemize}
\tightlist
\item
  As we expected, when \(n < p\), SVD decorrelation's result is not as
  good as full rank case ( \(n\geq p\)), which means off diagonal
  elements are not equal or closed to zero
\item
  By just looking at the final correlation matrix,
\end{itemize}


\end{document}
