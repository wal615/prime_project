\documentclass[]{article}
\usepackage{lmodern}
\usepackage{amssymb,amsmath}
\usepackage{ifxetex,ifluatex}
\usepackage{fixltx2e} % provides \textsubscript
\ifnum 0\ifxetex 1\fi\ifluatex 1\fi=0 % if pdftex
  \usepackage[T1]{fontenc}
  \usepackage[utf8]{inputenc}
\else % if luatex or xelatex
  \ifxetex
    \usepackage{mathspec}
  \else
    \usepackage{fontspec}
  \fi
  \defaultfontfeatures{Ligatures=TeX,Scale=MatchLowercase}
\fi
% use upquote if available, for straight quotes in verbatim environments
\IfFileExists{upquote.sty}{\usepackage{upquote}}{}
% use microtype if available
\IfFileExists{microtype.sty}{%
\usepackage{microtype}
\UseMicrotypeSet[protrusion]{basicmath} % disable protrusion for tt fonts
}{}
\usepackage[margin=1in]{geometry}
\usepackage{hyperref}
\hypersetup{unicode=true,
            pdftitle={SVD Dimension reduction method},
            pdfauthor={Xuelong Wang},
            pdfborder={0 0 0},
            breaklinks=true}
\urlstyle{same}  % don't use monospace font for urls
\usepackage{graphicx,grffile}
\makeatletter
\def\maxwidth{\ifdim\Gin@nat@width>\linewidth\linewidth\else\Gin@nat@width\fi}
\def\maxheight{\ifdim\Gin@nat@height>\textheight\textheight\else\Gin@nat@height\fi}
\makeatother
% Scale images if necessary, so that they will not overflow the page
% margins by default, and it is still possible to overwrite the defaults
% using explicit options in \includegraphics[width, height, ...]{}
\setkeys{Gin}{width=\maxwidth,height=\maxheight,keepaspectratio}
\IfFileExists{parskip.sty}{%
\usepackage{parskip}
}{% else
\setlength{\parindent}{0pt}
\setlength{\parskip}{6pt plus 2pt minus 1pt}
}
\setlength{\emergencystretch}{3em}  % prevent overfull lines
\providecommand{\tightlist}{%
  \setlength{\itemsep}{0pt}\setlength{\parskip}{0pt}}
\setcounter{secnumdepth}{5}
% Redefines (sub)paragraphs to behave more like sections
\ifx\paragraph\undefined\else
\let\oldparagraph\paragraph
\renewcommand{\paragraph}[1]{\oldparagraph{#1}\mbox{}}
\fi
\ifx\subparagraph\undefined\else
\let\oldsubparagraph\subparagraph
\renewcommand{\subparagraph}[1]{\oldsubparagraph{#1}\mbox{}}
\fi

%%% Use protect on footnotes to avoid problems with footnotes in titles
\let\rmarkdownfootnote\footnote%
\def\footnote{\protect\rmarkdownfootnote}

%%% Change title format to be more compact
\usepackage{titling}

% Create subtitle command for use in maketitle
\newcommand{\subtitle}[1]{
  \posttitle{
    \begin{center}\large#1\end{center}
    }
}

\setlength{\droptitle}{-2em}

  \title{SVD Dimension reduction method}
    \pretitle{\vspace{\droptitle}\centering\huge}
  \posttitle{\par}
    \author{Xuelong Wang}
    \preauthor{\centering\large\emph}
  \postauthor{\par}
      \predate{\centering\large\emph}
  \postdate{\par}
    \date{2018-11-15}

\usepackage{float,amsmath, bbm, siunitx, bm}
\usepackage{pdfpages}
\floatplacement{figure}{H}
\newcommand{\indep}{\rotatebox[origin=c]{90}{$\models$}}

\begin{document}
\maketitle

{
\setcounter{tocdepth}{2}
\tableofcontents
}
\section{Motivation}\label{motivation}

Based on previous simulation results we did a series of simulation on
estimation of total variance of main and interactive effects. we found
that combing dimension reduction with decorrelation tend (our proposed
method) to have a better result than GCTA, especailly when n \textless{}
p and correlation between covariates are high. Therefore, we condcuted a
group of simulation studies trying to evaluate the performance of the
proposed method. we tried different covariance structures and PCBs data
with re-sampling. Overall, the performance is good in most of the case.
When n is small and correlation is also weak, the prospoed method is as
good as the original GCTA method.

\section{Main idea two steps}\label{main-idea-two-steps}

\subsection{Dimension Reduction}\label{dimension-reduction}

\begin{align*}
  X = U D V^T &= \begin{bmatrix}
                      U_r & U_2
                      \end{bmatrix}
                      \begin{bmatrix}
                      D_r & 0\\
                      0 & D_2
                      \end{bmatrix}
                      \begin{bmatrix}
                      V_r & V_2\\
                      V_3 & V_4
                      \end{bmatrix}^T \\ 
              &= 
                      \begin{bmatrix}
                      U_rD_r & U_2D_2
                      \end{bmatrix}
                      \begin{bmatrix}
                      V_r^T & V_3^T\\
                      V_2^T & V_4^T
                      \end{bmatrix}
                      =
                      \begin{bmatrix}
                      U_rD_rV_r^T + U_2D_2V_2^T & U_rD_rV_3^T + U_2D_2 V_4^T
                      \end{bmatrix}
\end{align*}

Ignore \(V_2\), \(V_3\) and \(V_4\) , then we have the X\_r as following
\[
  X_r = U_rD_rV_r^T.
\] We use \(X_r\) as the new covariates to the proposed methd.
Therefore, we reduce the dimension from p to n

\subsection{Following with GCTA
method}\label{following-with-gcta-method}

After calculating \(X_r\), we can regard \(X_r\) as our new predictors
and use it as the input to the proposed method

Note that we could use this blocking method to reduce X's dimension to
\(k, k \leq min(p,n)\).

\section{Simulation study}\label{simulation-study}

I used Chi-square random variable with df = 1. To generate a certain
covariance structure, one could randomly generate a sample from
multivariate-normal-distribution first, and then just square each
elements to have a group univarate Chi-saure distribution with desired
correlations. The details of simulation is shown as follows.

\subsection{Simulation setup}\label{simulation-setup}

\begin{enumerate}
\def\labelenumi{\arabic{enumi}.}
\item
  Normal distribution\\
  \[
  X = [X_1 \dots, X_p] ~~~ cov(X_i, X_j) = \Sigma_{X}
    \]
\item
  Chi-square distribution\\
  \[
  T = [T_1 \dots, T_p] ,~~~ T_i = X_i^2 \sim \chi_{(1)}^2, ~~~ cov(T_i, T_j) = \Sigma_{\chi^2}
    \]
\end{enumerate}

\begin{itemize}
\tightlist
\item
  The sample size n is from 100 to 800
\item
  The number of main effect is 34 (p = 34)
\end{itemize}

\subsubsection{\texorpdfstring{correlation of \(T_i\) and
\(T_j\)}{correlation of T\_i and T\_j}}\label{correlation-of-t_i-and-t_j}

Assume \(Cov(X_i,X_j) = \sigma_{ij}, ~~ Var(X_i) = \sigma_i^2\),
\(E(X_i) = 0\) and constant variance, then we have \[
  Var(X_i) = E(X_i^2) - E(X_i)^2 = E(X_i^2) = \sigma_i^2 = \sigma^2
\]

\begin{align*}
  Cov(T_i, T_j) = Cov(X_i^2, X_i^2) &= E\left((X_i^2 - E(X_i^2))(X_j^2 - E(X_j^2))\right) \\ 
                                    &= E(X_i^2X_j^2 - X_i^2E(X_j^2) - X_j^2E(X_i^2) + E(X_i^2)E(X_j^2)) \\
                                    &= E(X_i^2X_j^2) - \sigma^4 \\ 
                                    &= \sigma_i^2\sigma_j^2 + 2\sigma_{ij}^2 - \sigma^4 \\ 
                                    &= 2\sigma^2_{ij}
\end{align*}

\subsubsection{Compound Symmetry}\label{compound-symmetry}

\[
  T = [T_1 \dots, T_p] ,~~~ T_i \sim \chi_{(1)}^2, ~~~ cov(T_i, T_j) = 2\rho^2,~~~ \forall  i \ne j, \rho = \{0.1, \dots, 0.9 \} 
\]

\subsubsection{Autoregression AR(1)}\label{autoregression-ar1}

\[
  T = [T_1 \dots, T_p] ,~~~ T_i \sim \chi_{(1)}^2, ~~~ cov(T_i, T_j) = 2\rho^{2|i-j|},~~~ \forall  i \ne j, \rho = \{0.1, \dots, 0.9 \} 
\]

\subsubsection{Unstructure}\label{unstructure}

\[
  T = [T_1 \dots, T_p] ,~~~ T_i \sim \chi_{(1)}^2, ~~~ cov(T_i, T_j) = \sigma_{ij}
\]

\subsection{Simulation result}\label{simulation-result}

\subsubsection{Compound Symmetry}\label{compound-symmetry-1}

\subsubsection{Autoregression}\label{autoregression}

\subsubsection{Unstructure}\label{unstructure-1}


\end{document}
