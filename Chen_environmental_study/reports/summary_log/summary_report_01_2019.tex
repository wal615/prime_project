\documentclass[]{article}
\usepackage{lmodern}
\usepackage{amssymb,amsmath}
\usepackage{ifxetex,ifluatex}
\usepackage{fixltx2e} % provides \textsubscript
\ifnum 0\ifxetex 1\fi\ifluatex 1\fi=0 % if pdftex
  \usepackage[T1]{fontenc}
  \usepackage[utf8]{inputenc}
\else % if luatex or xelatex
  \ifxetex
    \usepackage{mathspec}
  \else
    \usepackage{fontspec}
  \fi
  \defaultfontfeatures{Ligatures=TeX,Scale=MatchLowercase}
\fi
% use upquote if available, for straight quotes in verbatim environments
\IfFileExists{upquote.sty}{\usepackage{upquote}}{}
% use microtype if available
\IfFileExists{microtype.sty}{%
\usepackage{microtype}
\UseMicrotypeSet[protrusion]{basicmath} % disable protrusion for tt fonts
}{}
\usepackage[margin=1in]{geometry}
\usepackage{hyperref}
\hypersetup{unicode=true,
            pdftitle={Summary report about proposed method},
            pdfauthor={Xuelong Wang},
            pdfborder={0 0 0},
            breaklinks=true}
\urlstyle{same}  % don't use monospace font for urls
\usepackage{graphicx,grffile}
\makeatletter
\def\maxwidth{\ifdim\Gin@nat@width>\linewidth\linewidth\else\Gin@nat@width\fi}
\def\maxheight{\ifdim\Gin@nat@height>\textheight\textheight\else\Gin@nat@height\fi}
\makeatother
% Scale images if necessary, so that they will not overflow the page
% margins by default, and it is still possible to overwrite the defaults
% using explicit options in \includegraphics[width, height, ...]{}
\setkeys{Gin}{width=\maxwidth,height=\maxheight,keepaspectratio}
\IfFileExists{parskip.sty}{%
\usepackage{parskip}
}{% else
\setlength{\parindent}{0pt}
\setlength{\parskip}{6pt plus 2pt minus 1pt}
}
\setlength{\emergencystretch}{3em}  % prevent overfull lines
\providecommand{\tightlist}{%
  \setlength{\itemsep}{0pt}\setlength{\parskip}{0pt}}
\setcounter{secnumdepth}{5}
% Redefines (sub)paragraphs to behave more like sections
\ifx\paragraph\undefined\else
\let\oldparagraph\paragraph
\renewcommand{\paragraph}[1]{\oldparagraph{#1}\mbox{}}
\fi
\ifx\subparagraph\undefined\else
\let\oldsubparagraph\subparagraph
\renewcommand{\subparagraph}[1]{\oldsubparagraph{#1}\mbox{}}
\fi

%%% Use protect on footnotes to avoid problems with footnotes in titles
\let\rmarkdownfootnote\footnote%
\def\footnote{\protect\rmarkdownfootnote}

%%% Change title format to be more compact
\usepackage{titling}

% Create subtitle command for use in maketitle
\newcommand{\subtitle}[1]{
  \posttitle{
    \begin{center}\large#1\end{center}
    }
}

\setlength{\droptitle}{-2em}

  \title{Summary report about proposed method}
    \pretitle{\vspace{\droptitle}\centering\huge}
  \posttitle{\par}
    \author{Xuelong Wang}
    \preauthor{\centering\large\emph}
  \postauthor{\par}
      \predate{\centering\large\emph}
  \postdate{\par}
    \date{2019-01-11}

\usepackage{float,amsmath, bbm, siunitx, bm}
\floatplacement{figure}{H}
\newcommand{\indep}{\rotatebox[origin=c]{90}{$\models$}}

\begin{document}
\maketitle

{
\setcounter{tocdepth}{2}
\tableofcontents
}
\section{Topic}\label{topic}

The overall goal of this project is to understand the relationships
among chemical exposures and health outcomes. Since the relation could
be very complicated and the effect of each chemical factor could very
weak, one may want to model the relation of variance between chemical
factors and health outcome.

To achieve that goal we need to break things into steps, so the current
goal of this project is to estimate the main and interactive effects
given simulated responses.

More specifically, we are trying to adopt and modify an approach called
GCTA method, which is used for estimating of heritablity in genoew-wide
study.

\section{Model}\label{model}

The model we are using is mixed model with main effect and interaction
effect. The effects could either be fixed or random effect. But for now,
we assume that both of them are fixed.

\[
  y = \alpha + \sum_{j = 1}^p x_j\beta_j + \sum_{j \neq k} \gamma_{jk}x_jx_k + \epsilon.
\] Matrix form \[
  y = X^T\beta + X^T\Gamma X + \epsilon,
\] Where

\begin{itemize}
\tightlist
\item
  \(X = (x_{1}, \dots x_{p})^T\), in our case assumpe
  \(X \sim N(0, \Sigma_p)\)\\
\item
  \(\epsilon \indep x_{ji}\)\\
\item
  \(\beta = (\beta_1, \dots, \beta_p)^T\) is fixed\\
\item
  \(\Gamma\) is a \(p \times p\) matrix with diagonal elements equal to
  0.
\end{itemize}

\section{GCTA and proposed method (a modified GCTA
method)}\label{gcta-and-proposed-method-a-modified-gcta-method}

The details of the GCTA and proposed method could be found in previous
report (simulation of fixed and random effect). The main idea of the
proposed method is to add a decorrelation step, so that the GCTA method
could deal with correlated data.

There is a suggestion (Aim 1(b) Proposal) of GCTA method. In order to
let the method work correctly, the causal covariates to be independent
themselves and independent of non-causal covariates. But based on the
simulation study and some theoretical results, we found that as long as
the main effect and the interaction effect are uncorrelated to each
other, \(\mathbf{Cov(X_m^T\beta_{m}, X_i^T\beta_i)}\), then we are able
to estimate both of the effects' variance without much bias. This
suggests that the \textbf{Independence} of covariates may not be that
crucial.

However, if the correlation between main and interaction is not zero,
then it will cause some trouble in variances estimation. The correlation
term, which is not considered by the GCTA method, will affect the
estimation result for both effect. One solution for walking round that
problem is a two-step method. Firstly, we estimate the total variance,
which is the summation of main and interaction and their correlation
method. And then, we use some statistical test to determine if there
exits an interaction effect. Followings are some details of the methods.

\section{Variance Estimation under different
conditions}\label{variance-estimation-under-different-conditions}

Before we go into details, Let me just rewrite the issue part in math
formula so that we could get a better understand.

\begin{align*}
Var(Y) &= Var(X^T\beta + X^T\Gamma X) + Var(\epsilon) \\
         &= Var(X^T\beta) + Var(X^T\Gamma X) + 2Cov(X^T\beta, X^T\Gamma X) + Var(\epsilon) \\
\end{align*}

\begin{enumerate}
\def\labelenumi{\arabic{enumi}.}
\tightlist
\item
  There is an additional terms \(Cov(X^T\beta, X^T\Gamma X)\)
\item
  The main effect \(x_i\) and the interaction effect \(x_ix_j\) are
  dependent and cannot be independent anymore. Besides, even if \(X\) is
  an independent random vector, the interaction effect are not
  independent themselves, i.g. \(x_i x_j\) and \(X X_{j'}\) are
  dependent.
\item
  In order to keep the variance structure, we can only apply the linear
  transformation on the main effects, not the interactive effects.
\end{enumerate}

\subsection{Normal distribution}\label{normal-distribution}

Let's just start with the most straightforward one, which is when
covariates follows a Normal distribution. The properties of normal
distribution simplify the situation, so that the proposed method can
work well. Namely, no matter covariates are independent or not, we can
always have

\begin{align*}
Cov(X_i^T\beta, X_i^T\Gamma X_i) &= E[(X_i^T\beta - E(X_i^T\beta))(X_i^T\Gamma X_i - E(X_i^T\Gamma X_i))]\\
    &= E[X_i^T\beta(X_i^T\Gamma X_i - E(X_i^T\Gamma X_i)) \\
    &= E[X_i^T\beta(X_i^T\Gamma X_i - trace(\Gamma\Sigma_p))] \tag*{Note that $\gamma_{jj} = 0$}\\
    &= E[X_i^T\beta \cdot X_i^T\Gamma X_i] \\
    &= E[(\sum_m(x_{im}\beta_m))(\sum_j\sum_k\gamma_{jk}x_{ij} x_{ik}) ] \\
    & = 0 \tag*{Note that $E(x_i^2x_j) = 0$ because of centered data}
\end{align*}

Then we have,

\begin{align*}
Var(Y) &= Var(X^T\beta + X^T\Gamma X) + Var(\epsilon) \\
         &= Var(X^T\beta) + Var(X^T\Gamma X) + Var(\epsilon) \\
\end{align*}

. Therefore, we don't have to worry about the covariance term and should
expect good variance estimation result

Following are some results and conclusion based on simulation and
theoretical study.

\subsubsection{Independent covariates}\label{independent-covariates}

In this situation, both GCTA and proposed method can work well.

\subsubsection{Dependent covariates}\label{dependent-covariates}

In this situation, the proposed method's performance is much better then
the GCTA itself in term of unbiaseness. This indicates that the
correlation structure may be more necessary for the GCTA method than
independency. The simulation result could be found on my report with
date 08/15/2018

The normal distribution is relatively easy to deal with because of the
property. We could just add an decorrelation step before the GCTA
method, than we could get a good result. This also indicates an option
to deal with a more complicated problem, for example, the non-normal
distribution. We could just transfer the data into a normal or
normal-like distribution and do the analysis as usual. But we could not
or hard to control the magnitude of main and interaction variance and
their relative relation after transfermation.

\subsection{Non-normal distribution}\label{non-normal-distribution}

\subsubsection{Independent covariates}\label{independent-covariates-1}

For Independent case, GCTA method appears to work fine.

\subsubsection{Dependent covariates}\label{dependent-covariates-1}

Now, we move to a more general and also more complicated situation,
non-normal distribution. For the non-normal or long tail distribution,
we cannot guarantee that the third monment equal to zero, therefore the
covariance of main and interaction effect is no longer zero. Simulation
study (08/15/2018) has shown that in such case, even the proposed method
cannot estimate both of the effect correctly.

\subsection{sub sampling method for evaluating methods'
performance}\label{sub-sampling-method-for-evaluating-methods-performance}


\end{document}
