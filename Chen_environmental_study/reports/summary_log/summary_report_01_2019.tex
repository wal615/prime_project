\documentclass[]{article}
\usepackage{lmodern}
\usepackage{amssymb,amsmath}
\usepackage{ifxetex,ifluatex}
\usepackage{fixltx2e} % provides \textsubscript
\ifnum 0\ifxetex 1\fi\ifluatex 1\fi=0 % if pdftex
  \usepackage[T1]{fontenc}
  \usepackage[utf8]{inputenc}
\else % if luatex or xelatex
  \ifxetex
    \usepackage{mathspec}
  \else
    \usepackage{fontspec}
  \fi
  \defaultfontfeatures{Ligatures=TeX,Scale=MatchLowercase}
\fi
% use upquote if available, for straight quotes in verbatim environments
\IfFileExists{upquote.sty}{\usepackage{upquote}}{}
% use microtype if available
\IfFileExists{microtype.sty}{%
\usepackage{microtype}
\UseMicrotypeSet[protrusion]{basicmath} % disable protrusion for tt fonts
}{}
\usepackage[margin=1in]{geometry}
\usepackage{hyperref}
\hypersetup{unicode=true,
            pdftitle={Summary report about proposed method},
            pdfauthor={Xuelong Wang},
            pdfborder={0 0 0},
            breaklinks=true}
\urlstyle{same}  % don't use monospace font for urls
\usepackage{graphicx,grffile}
\makeatletter
\def\maxwidth{\ifdim\Gin@nat@width>\linewidth\linewidth\else\Gin@nat@width\fi}
\def\maxheight{\ifdim\Gin@nat@height>\textheight\textheight\else\Gin@nat@height\fi}
\makeatother
% Scale images if necessary, so that they will not overflow the page
% margins by default, and it is still possible to overwrite the defaults
% using explicit options in \includegraphics[width, height, ...]{}
\setkeys{Gin}{width=\maxwidth,height=\maxheight,keepaspectratio}
\IfFileExists{parskip.sty}{%
\usepackage{parskip}
}{% else
\setlength{\parindent}{0pt}
\setlength{\parskip}{6pt plus 2pt minus 1pt}
}
\setlength{\emergencystretch}{3em}  % prevent overfull lines
\providecommand{\tightlist}{%
  \setlength{\itemsep}{0pt}\setlength{\parskip}{0pt}}
\setcounter{secnumdepth}{5}
% Redefines (sub)paragraphs to behave more like sections
\ifx\paragraph\undefined\else
\let\oldparagraph\paragraph
\renewcommand{\paragraph}[1]{\oldparagraph{#1}\mbox{}}
\fi
\ifx\subparagraph\undefined\else
\let\oldsubparagraph\subparagraph
\renewcommand{\subparagraph}[1]{\oldsubparagraph{#1}\mbox{}}
\fi

%%% Use protect on footnotes to avoid problems with footnotes in titles
\let\rmarkdownfootnote\footnote%
\def\footnote{\protect\rmarkdownfootnote}

%%% Change title format to be more compact
\usepackage{titling}

% Create subtitle command for use in maketitle
\newcommand{\subtitle}[1]{
  \posttitle{
    \begin{center}\large#1\end{center}
    }
}

\setlength{\droptitle}{-2em}

  \title{Summary report about proposed method}
    \pretitle{\vspace{\droptitle}\centering\huge}
  \posttitle{\par}
    \author{Xuelong Wang}
    \preauthor{\centering\large\emph}
  \postauthor{\par}
      \predate{\centering\large\emph}
  \postdate{\par}
    \date{2019-01-18}

\usepackage{float,amsmath, bbm, siunitx, bm}
\floatplacement{figure}{H}
\newcommand{\indep}{\rotatebox[origin=c]{90}{$\models$}}

\begin{document}
\maketitle

{
\setcounter{tocdepth}{2}
\tableofcontents
}
\section{Topic}\label{topic}

The overall goal of this project is to understand the relationships
among chemical exposures and health outcomes. Since the relation could
be very complicated and the effect of each chemical factor could very
weak, one may want to model the relation of variance between chemical
factors and health outcome.

To achieve that goal we need to break things into steps, so the current
goal of this project is to estimate the main and interactive effects
given simulated responses.

More specifically, we are trying to adopt and modify an approach called
GCTA method, which is used for estimating of heritablity in genoew-wide
study.

\section{Model}\label{model}

The model we are using is mixed model with main effect and interaction
effect. The effects could either be fixed or random effect. But for now,
we assume that both of them are fixed.

\[
  y = \alpha + \sum_{j = 1}^p x_j\beta_j + \sum_{j \neq k} \gamma_{jk}x_jx_k + \epsilon.
\] Matrix form \[
  y = X^T\beta + X^T\Gamma X + \epsilon,
\] Where

\begin{itemize}
\tightlist
\item
  \(X = (x_{1}, \dots x_{p})^T\), in our case assumpe
  \(X \sim N(0, \Sigma_p)\)\\
\item
  \(\epsilon \indep x_{ji}\)\\
\item
  \(\beta = (\beta_1, \dots, \beta_p)^T\) is fixed\\
\item
  \(\Gamma\) is a \(p \times p\) matrix with diagonal elements equal to
  0.
\end{itemize}

\section{GCTA and proposed method (a modified GCTA
method)}\label{gcta-and-proposed-method-a-modified-gcta-method}

The details of the GCTA and proposed method could be found in previous
report (simulation of fixed and random effect). The main idea of the
proposed method is to add a decorrelation step, so that the GCTA method
could deal with correlated data.

There is a suggestion (Aim 1(b) Proposal) of GCTA method. In order to
let the method work correctly, the causal covariates to be independent
themselves and independent of non-causal covariates. But based on the
simulation study and some theoretical results, we found that as long as
the main effect and the interaction effect are uncorrelated to each
other, \(\mathbf{Cov(X_m^T\beta_{m}, X_i^T\beta_i)}\), then we are able
to estimate both of the effects' variance without much bias. This
suggests that the \textbf{Independence} of covariates may not be that
crucial.

However, if the correlation between main and interaction is not zero,
then it will cause some trouble in variances estimation. The correlation
term, which is not considered by the GCTA method, will affect the
estimation result for both effect. One solution for walking round that
problem is a two-step method. Firstly, we estimate the total variance,
which is the summation of main and interaction and their correlation
method. And then, we use some statistical test to determine if there
exits an interaction effect. Followings are some details of the methods.

\subsection{the proposed method}\label{the-proposed-method}

The approach of the proposed method is to uncorrelated the observed
covariates by using a linear transformation, i.e. \(Z = A^{-1}X\), So
that \(Var(Z) = I_p\). In this way, we could get the uncorrelated
predictors and use them as the input of the regular GCTA method.

\section{Variance Estimation under different
conditions}\label{variance-estimation-under-different-conditions}

Before we go into details, Let me just rewrite the issue part in math
formula so that we could get a better understand.

\begin{align*}
Var(Y) &= Var(X^T\beta + X^T\Gamma X) + Var(\epsilon) \\
         &= Var(X^T\beta) + Var(X^T\Gamma X) + 2Cov(X^T\beta, X^T\Gamma X) + Var(\epsilon) \\
\end{align*}

\begin{enumerate}
\def\labelenumi{\arabic{enumi}.}
\tightlist
\item
  There is an additional terms \(Cov(X^T\beta, X^T\Gamma X)\)
\item
  The main effect \(x_i\) and the interaction effect \(x_ix_j\) are
  dependent and cannot be independent anymore. Besides, even if \(X\) is
  an independent random vector, the interaction effect are not
  independent themselves, i.g. \(x_i x_j\) and \(X X_{j'}\) are
  dependent.
\item
  In order to keep the variance structure, we can only apply the linear
  transformation on the main effects, not the interactive effects.
\end{enumerate}

\subsection{Normal distribution}\label{normal-distribution}

\subsubsection{Independent covariates}\label{independent-covariates}

In this situation, both GCTA and proposed method can work well.\\
Let's just start with the most straightforward one, which is when
covariates follows a Normal distribution. The properties of normal
distribution simplify the situation, so that the proposed method can
work well. Namely, no matter covariates are independent or not, we can
always have

\begin{align*}
Cov(X^T\beta, X^T\Gamma X) &= E[(X^T\beta - E(X^T\beta))(X^T\Gamma X - E(X^T\Gamma X))]\\
    &= E[X^T\beta(X^T\Gamma X - E(X^T\Gamma X)) \\
    &= E[X^T\beta(X^T\Gamma X - trace(\Gamma\Sigma_p))] \tag*{Note that $\gamma_{jj} = 0$}\\
    &= E[X^T\beta \cdot X^T\Gamma X] \\
    &= E[(\sum_m(x_{m}\beta_m))(\sum_j\sum_k\gamma_{jk}x_{j} x_{k}) ] \\
    & = 0 \tag*{Note that $E(x^2x_j) = 0$ because of centered data}
\end{align*}

Then we have,

\begin{align*}
Var(Y) &= Var(X^T\beta + X^T\Gamma X) + Var(\epsilon) \\
         &= Var(X^T\beta) + Var(X^T\Gamma X) + Var(\epsilon) \\
\end{align*}

. Therefore, we don't have to worry about the covariance term and should
expect good variance estimation result

\subsubsection{Dependent covariates}\label{dependent-covariates}

In this situation, the proposed method's performance is much better then
the GCTA itself in term of unbiaseness. This indicates that the
correlation structure may be more necessary for the GCTA method than
independency. But because the covariates follows normal distribution,
which means that after decorrelation, they become independent variables.
The simulation result could be found on my report with date 08/15/2018

Let's assume \(X \sim N(\mu, \Sigma)\), after standardization we have \[
  E(X) = 0 ~~ Var(X) = \tilde{\Sigma},
\] Where \(diag(\tilde{\Sigma}) = \vec{1}\), and
\(\tilde{\Sigma}_{ij} = \rho_{ij}\).

If we uncorrelated the X by \(Z = A^{-1}X\), where
\(A = \tilde{\Sigma}^{1/2}\), then we will have

\[
  Z \sim N(0,I_p), 
\] We are back to the independent case again.

A thougth for take advantage of the properties of normal distirbution:
The normal distribution is relatively easy to deal with because of the
property. We could just add a decorrelation step before the GCTA method,
than we could get a good result. This also indicates an option to deal
with a more complicated problem, for example, the non-normal
distribution. We could just transfer the data into a normal or
normal-like distribution and do the analysis as usual. But we could not
or hard to control the magnitude of main and interaction variance and
their relative relation after transfermation.

\subsection{Non-normal distribution}\label{non-normal-distribution}

\subsubsection{Independent covariates}\label{independent-covariates-1}

For Independent case, GCTA method appears to work fine. Similar with the
independent normal distribution, the covariance of main and interaction
effect will be zero. One condition is that we need the covariates to be
standardized.

\subsubsection{Dependent covariates}\label{dependent-covariates-1}

Now, we move to a more general and also more complicated situation,
non-normal distribution. For the non-normal or long tail distribution,
we cannot make them indepedent by just some linear operation, therefore
the covariance of main and interaction effect is no longer zero.
Simulation study (08/15/2018) has shown that in such case, even the
proposed method cannot estimate both of the effects correctly.

However, we can take one step back, which means instead of considering
the main and interaction effect separately, we look at the total
variance related to covariates:

\begin{align*}
  Var(Y) &= Var(X^T\beta + X^T \Gamma X) + Var(\epsilon) \\
           &= Var(X^T\beta + X_{inter}^T \gamma) + Var(\epsilon) \\
           &= \begin{bmatrix}   
                  X \\        
                  X_{inter}   
              \end{bmatrix}^T     
              \begin{bmatrix}
                \beta \\
                \gamma
              \end{bmatrix} + Var(\epsilon) \\
           &= Var(X_{total}^T \beta_{total}) + Var(\epsilon)
\end{align*}

In this way, we combine the main and interaction effect together as the
new covariates. The good news for estimation of the total variance is we
don't need to worry about the covariance part, because it is always
included in the total variance term. The downside is that we will never
make the total effect to be independent themselves anymore, because the
interaction terms contains the main effects.

Based on the simulation, we find the proposed method works fine, and it
works much better than the origianl GCTA method. After decorrelation,
the GCTA is able to estimate the total variance accuretly based on the
simualtion result.

\subsection{Decorrelation method}\label{decorrelation-method}

For Decoorelation step, it could be done by using the Spectral
decomposition if the observed covariates X is full column rank. After
adding this decorrelation step, the GCTA can work well in term of
estimation the main and interaction variance. However, if \(p > n\),
which means the number of parameter is larger than the number of
observation, so the sample covariance is not full rank, then the
proposed method suddenly work much worse. One working solution for that
issuse is to reduce the dimension of covariate and then apply the
decorrealtion.

\subsubsection{Spectral decomposition}\label{spectral-decomposition}

\[
  Var(X) = \Sigma_X = U\Lambda U^T,
\]

\begin{itemize}
\tightlist
\item
  \(X\) is the random vector with dim as \(p \times 1\),\\
\item
  \(\Sigma_X\) is \(p \times p\) symmetry and p.d. matrix,\\
\item
  \(\Lambda\) is a diagonal matrix with each diagonal element as the
  eigenvalue.
\end{itemize}

\subsubsection{\texorpdfstring{Assume the \(\Sigma_X\) is full
rank}{Assume the \textbackslash{}Sigma\_X is full rank}}\label{assume-the-sigma_x-is-full-rank}

To decorreate the X, we could just take the reciprocal of each square
root of eigenvalue as following.

\[
  \Sigma^{-\frac{1}{2}}_X = U\Lambda^{-\frac{1}{2}}U^T,
\] where
\(\Lambda^{-\frac{1}{2}} = \begin{bmatrix}  e_1^{-\frac{1}{2}} & \dots & 0 \\  \vdots & \ddots & \vdots \\  0 & \dots & e_p^{-\frac{1}{2}}  \end{bmatrix}\)

So that after transformation the \(\Sigma^{-\frac{1}{2}}_X X\) has
identity covariance matrix as following,

\[
  Var(\Sigma^{-\frac{1}{2}}_X X) = \Sigma^{-\frac{1}{2}}_X \Sigma_X\Sigma^{-\frac{1}{2}}_X = U\Lambda^{-\frac{1}{2}}U^T U\Lambda^{-1}U^T U\Lambda^{-\frac{1}{2}}U^T = I_p.
\]

\paragraph{\texorpdfstring{Assume the \(\Sigma_X\) is not full
rank}{Assume the \textbackslash{}Sigma\_X is not full rank}}\label{assume-the-sigma_x-is-not-full-rank}

\[
  Var(X) = \Sigma_X = U\Lambda U^T =
                        \begin{bmatrix}
                         U_1 & U_2\\
                        \end{bmatrix}
                        \begin{bmatrix}
                        \Lambda_1 & 0\\
                        0 & 0
                        \end{bmatrix}
                        \begin{bmatrix}
                        U_1^T \\
                        U_2^T
                        \end{bmatrix} = U_1\Lambda_1U_1^T,
\] - \(U_1\) is a \(p \times r\) matrix with r \textless{} p and in most
of case r = n the sample size.

Then after applying the same procedure we get following,

\[
  \Sigma^{-\frac{1}{2}}_X = U_1\Lambda_1^{-\frac{1}{2}}U_1^T,
\] Note that in this case, I'm using Moore Penrose inverse.

After transformation the X we have, \[
  Var(\Sigma^{-\frac{1}{2}}_X X) = \Sigma^{-\frac{1}{2}}_X \Sigma_X\Sigma^{-\frac{1}{2}}_X = U_1\Lambda^{-\frac{1}{2}}_1U^T_1 U_1\Lambda^{-1}_1U^T_1 U_1\Lambda^{-\frac{1}{2}}_1U^T_1 = U_1U_1^T,  
\] Note that by the property of the U we have \[
  U_1U_1^T + U_2U_2^T = I_p, ~~~~~\\
  (U_1U_1^T)^T U_1U_1^T = U_1U_1^T,
\] Besides, \(U_1U_1^T\) and \(U_2U_2^T\) are indempotent and
\(rank(U_2U_2^T) + rank(U_1U_1^T) = p\).

So if the X is not full rank we cannot decorrelation the covariance
matrix to an identity matrix.

\subsubsection{SVD with dimension reduction
step}\label{svd-with-dimension-reduction-step}

One possible solution for the issue of too many predictors is to reduce
the column dimension of covariates first and then used the reduced
covariate as the input for the prosposed method. I have tried several
dimension reduction method, and found that a modified SVD method
provides promosing results based on simulation study. the details as
following:

\begin{align*}
  X = U D V^T &= \begin{bmatrix}
                      U_r & U_2
                      \end{bmatrix}
                      \begin{bmatrix}
                      D_r & 0\\
                      0 & D_2
                      \end{bmatrix}
                      \begin{bmatrix}
                      V_r & V_2\\
                      V_3 & V_4
                      \end{bmatrix}^T \\ 
              &= 
                      \begin{bmatrix}
                      U_rD_r & U_2D_2
                      \end{bmatrix}
                      \begin{bmatrix}
                      V_r^T & V_3^T\\
                      V_2^T & V_4^T
                      \end{bmatrix}
                      =
                      \begin{bmatrix}
                      U_rD_rV_r^T + U_2D_2V_2^T & U_rD_rV_3^T + U_2D_2 V_4^T
                      \end{bmatrix}
\end{align*}

Ignore \(V_2\), \(V_3\) and \(V_4\) , then we have the X\_r as following
\[
  X_r = U_rD_rV_r^T.
\] We use \(X_r\) as the new covariates to the proposed methd.
Therefore, we reduce the dimension from p to n. After calculating
\(X_r\), we can regard \(X_r\) as our new predictors and use it as the
input to the proposed method Note that we could use this blocking method
to reduce X's dimension to \(k, k \leq min(p,n)\).

\subsubsection{PCA method}\label{pca-method}

\subsection{sub sampling method for evaluating methods'
performance}\label{sub-sampling-method-for-evaluating-methods-performance}


\end{document}
