\documentclass[]{article}
\usepackage{lmodern}
\usepackage{amssymb,amsmath}
\usepackage{ifxetex,ifluatex}
\usepackage{fixltx2e} % provides \textsubscript
\ifnum 0\ifxetex 1\fi\ifluatex 1\fi=0 % if pdftex
  \usepackage[T1]{fontenc}
  \usepackage[utf8]{inputenc}
\else % if luatex or xelatex
  \ifxetex
    \usepackage{mathspec}
  \else
    \usepackage{fontspec}
  \fi
  \defaultfontfeatures{Ligatures=TeX,Scale=MatchLowercase}
\fi
% use upquote if available, for straight quotes in verbatim environments
\IfFileExists{upquote.sty}{\usepackage{upquote}}{}
% use microtype if available
\IfFileExists{microtype.sty}{%
\usepackage{microtype}
\UseMicrotypeSet[protrusion]{basicmath} % disable protrusion for tt fonts
}{}
\usepackage[margin=1in]{geometry}
\usepackage{hyperref}
\hypersetup{unicode=true,
            pdftitle={Simulation of fixed and random effect},
            pdfauthor={Xuelong Wang},
            pdfborder={0 0 0},
            breaklinks=true}
\urlstyle{same}  % don't use monospace font for urls
\usepackage{graphicx,grffile}
\makeatletter
\def\maxwidth{\ifdim\Gin@nat@width>\linewidth\linewidth\else\Gin@nat@width\fi}
\def\maxheight{\ifdim\Gin@nat@height>\textheight\textheight\else\Gin@nat@height\fi}
\makeatother
% Scale images if necessary, so that they will not overflow the page
% margins by default, and it is still possible to overwrite the defaults
% using explicit options in \includegraphics[width, height, ...]{}
\setkeys{Gin}{width=\maxwidth,height=\maxheight,keepaspectratio}
\IfFileExists{parskip.sty}{%
\usepackage{parskip}
}{% else
\setlength{\parindent}{0pt}
\setlength{\parskip}{6pt plus 2pt minus 1pt}
}
\setlength{\emergencystretch}{3em}  % prevent overfull lines
\providecommand{\tightlist}{%
  \setlength{\itemsep}{0pt}\setlength{\parskip}{0pt}}
\setcounter{secnumdepth}{5}
% Redefines (sub)paragraphs to behave more like sections
\ifx\paragraph\undefined\else
\let\oldparagraph\paragraph
\renewcommand{\paragraph}[1]{\oldparagraph{#1}\mbox{}}
\fi
\ifx\subparagraph\undefined\else
\let\oldsubparagraph\subparagraph
\renewcommand{\subparagraph}[1]{\oldsubparagraph{#1}\mbox{}}
\fi

%%% Use protect on footnotes to avoid problems with footnotes in titles
\let\rmarkdownfootnote\footnote%
\def\footnote{\protect\rmarkdownfootnote}

%%% Change title format to be more compact
\usepackage{titling}

% Create subtitle command for use in maketitle
\newcommand{\subtitle}[1]{
  \posttitle{
    \begin{center}\large#1\end{center}
    }
}

\setlength{\droptitle}{-2em}

  \title{Simulation of fixed and random effect}
    \pretitle{\vspace{\droptitle}\centering\huge}
  \posttitle{\par}
    \author{Xuelong Wang}
    \preauthor{\centering\large\emph}
  \postauthor{\par}
      \predate{\centering\large\emph}
  \postdate{\par}
    \date{2018-10-23}

\usepackage{float,amsmath, bbm, siunitx, bm}
\floatplacement{figure}{H}
\newcommand{\indep}{\rotatebox[origin=c]{90}{$\models$}}

\begin{document}
\maketitle

{
\setcounter{tocdepth}{2}
\tableofcontents
}
\section{Motivation}\label{motivation}

Based on the previous simulations results, we found that the proposed
method is able to work well on estimating the interactive effect in
certain situation. However, because of the interaction terms the
covariates cannot be \textbf{Independent} anymore. Thus, we may need to
conduct a series simulations with different setup in order to get a
large picture of what situation the proposed method can work well.

\section{Background}\label{background}

\subsection{GCTA method and proposed
method}\label{gcta-method-and-proposed-method}

In the GCTA method, the author suggests that the causal covariates
should be independent themselves and also be independent with non-causal
covariates. Therefore, to ensure the GCTA method is able to work
appropriately, we need to make sure the independence among the
covariates.

The approach of the proposed method is to uncorrelated the observed
covariates by using a linear transformation, i.e. \(Z = XA^{-1}\). Given
\(X \sim N(0, \Sigma)\), we can using the proposed method to get a
independent covariates. Thus, this method should work well when X's
columns are correlated to each other, which is also supported by
simulation results on the main effects with normal correlated
distributions.

How to reduce the correlation (the linear transformation) is important
part of the proposed method. Because we want to the decorrelated the
covariates so that they become independent. We don't want to pick up to
much noise during the decorrelation procedure. This is not the main
topic of this simulation, but we will do a more details study on the
decorrelation.

\subsection{Adding the interactive
terms}\label{adding-the-interactive-terms}

From the model point of view, it actually doesn't change a lot. We can
consider the interactive term as another part of covariates. However,
since the interactive effects have a strong relation with main effect,
that causes the estimation of variance of interactive effect
complicated.

\subsubsection{Issues}\label{issues}

\begin{align*}
Var(Y_i) &= Var(X_i^T\beta + X_i^T\Gamma X_i) + Var(\epsilon_i) \\
         &= Var(X_i^T\beta) + Var(X_i^T\Gamma X_i) + 2Cov(X_i^T\beta, X_i^T\Gamma X_i) + Var(\epsilon_i) \\
\end{align*}

\begin{enumerate}
\def\labelenumi{\arabic{enumi}.}
\tightlist
\item
  There is an additional terms \(Cov(X_i^T\beta, X_i^T\Gamma X_i)\)
\item
  The main effect \(X_i\) and \(X_i X_j\) are dependent and cannot be
  independent anymore. Besides, \(X_i X_j\) and \(X_i X_{j'}\) are
  dependent.
\item
  In order to keep the variance structure, we can only apply the linear
  transformation on the main effects, not the interactive effects.
\end{enumerate}

\subsubsection{Solutions}\label{solutions}

For the 1st issue, there are basic two different approach.

\begin{enumerate}
\def\labelenumi{\arabic{enumi}.}
\item
  If X or after some transformation follows normal distributions, then
  the \(Cov(Z_i^T\beta, Z_i^T\Gamma Z_i)\) will be zero. More
  specifically, as long as the 3rd moments are 0 the correlation of main
  and interactive effect should be zero.
\item
  If we assume \(\beta's\) or \(\Gamma's\) are independent random
  variables, then the correlation should also be round zero. Because we
  assume \(\beta \indep \Gamma\)
\end{enumerate}

For the 2nd issue, the simulation studies shows that the the proposed
method can work well under the normal distribution setup. So it suggests
that the Independence assumption may not be that restricted, at least
under the Normal distribution. One guess is that under normal
distribution the covariates of main and interactive is zero
\(Cov(X_i^T\beta, X_i^T\Gamma X_i) = 0\), which means they are
un-correlated. Although un-correlation can not guarantee independence,
it shows non-linear relation between main and interactive effects.

Thus, the main idea is to first transform the covariates into
normal-like distribution and applied the proposed method.

\section{Simulation}\label{simulation}

To get a better and larger picture of how well the proposed method's
estimations, we conduct a series simulation with different setup.

The main factors are

\begin{enumerate}
\def\labelenumi{\arabic{enumi}.}
\tightlist
\item
  Distribution of X\\
\item
  Independence or Dependence of X\\
\item
  Fixed or Random of main effect \(\beta's\)\\
\item
  Fixed or Random of interaction effect \(\Gamma's\)
\end{enumerate}

\subsection{Simulation procedure}\label{simulation-procedure}

\begin{figure}
\centering
\includegraphics{./simulation_workflow.png}
\caption{Simulation workflow}
\end{figure}

\newpage

\section{X follows Normal
distributions}\label{x-follows-normal-distributions}

If we can assume that X has normal distribution then the problem become
more straightforward, because the correlation of main and interactive
effects are zero. Besides, we don't need to transform X in order to make
it looks like a normal distribution.

\subsection{Independent}\label{independent}

Based on the previous simulation (from the Proposal), if X is normal and
independent, then both proposed and original GCTA method can estimate
the main and interaction effects well. This is indirectly suggest that
the dependence between main and interactive may not be a big issue.

\subsection{Dependent (correlated)}\label{dependent-correlated}

If the X's are normal but correlated with each other, then it's not easy
to estimate the interactive correctly. Based on previous simulation's
results, the proposed method can work much better than the original GCTA
when \(\beta\), the main effect, is fixed. However when the main
coefficients are not fixed then there is a big bias for the interaction
estimation of proposed method. A guess is that if the \(\beta\) is not
linear transformation of may affect the \(Var(X_i^T\Gamma X_i)\).

\newpage

\subsubsection{\texorpdfstring{\(\beta\) is fixed and \(\Gamma\) is
fixed}{\textbackslash{}beta is fixed and \textbackslash{}Gamma is fixed}}\label{beta-is-fixed-and-gamma-is-fixed}

The estimation and true values

\[
    Var(X_i^T\beta) = \beta^T \Sigma \beta
\]

\begin{align*}
    Var(X_i^T\Gamma X_i) = 2tr(\Gamma \Sigma \Gamma \Sigma) + 0
\end{align*}

Simulation result

\begin{figure}
\centering
\includegraphics{Simulation_report_files/figure-latex/main_fixed_fixed_normal-1.pdf}
\caption{Main Effect of normal fixed main and interaction}
\end{figure}

\begin{figure}
\centering
\includegraphics{Simulation_report_files/figure-latex/inter_fixed_fixed_normal-1.pdf}
\caption{Interactive Effect of normal fixed main and interaction}
\end{figure}

\clearpage

\subsubsection{\texorpdfstring{\(\beta\) is fixed and \(\Gamma\) is
random}{\textbackslash{}beta is fixed and \textbackslash{}Gamma is random}}\label{beta-is-fixed-and-gamma-is-random}

\[
    Var(X_i^T\beta) = \beta^T \Sigma \beta
\]

\begin{align*}
    Var(X_i^T\Gamma X_i) &= E(Var(X_i^T\Gamma X_i|X_i = x_i)) + Var(E(X_i^T\Gamma X_i|X_i = x_i)) \\ 
                         &= E(Var(X_i^T\Gamma X_i|X_i = x_i))
\end{align*}

Simulation result

\begin{figure}
\centering
\includegraphics{Simulation_report_files/figure-latex/main_fixed_random_normal-1.pdf}
\caption{Main Effect of normal fixed main and random interaction}
\end{figure}

\begin{figure}
\centering
\includegraphics{Simulation_report_files/figure-latex/inter_fixed_random_normal-1.pdf}
\caption{Interactive Effect of normal fixed main and random interaction}
\end{figure}

\clearpage

\subsubsection{\texorpdfstring{\(\beta\) is random and \(\Gamma\) is
random}{\textbackslash{}beta is random and \textbackslash{}Gamma is random}}\label{beta-is-random-and-gamma-is-random}

\begin{align*}
    Var(X_i^T\beta) &= E(Var(X_i^T\beta|X_i = x_i)) + Var(E(X_i^T\beta|X_i = x_i))\\
                    &= E(Var(X_i^T\beta|X_i = x_i)) + 0 \\
                    &= E(X_i^T \Sigma_{\beta} X_i)  \tag*{If $\Sigma_{\beta} = I_p$}\\
                    &= E(X_i^TX_i)
\end{align*}

\begin{align*}
    Var(X_i^T\Gamma X_i) &= E(Var(X_i^T\Gamma X_i|X_i = x_i)) + Var(E(X_i^T\Gamma X_i|X_i = x_i)) \\ 
                         &= E(Var(X_i^T\Gamma X_i|X_i = x_i))
\end{align*}

\begin{figure}
\centering
\includegraphics{Simulation_report_files/figure-latex/main_random_random_normal-1.pdf}
\caption{Main Effect of normal random main and interaction}
\end{figure}

\begin{figure}
\centering
\includegraphics{Simulation_report_files/figure-latex/inter_random_random_normal-1.pdf}
\caption{Interactive Effect of normal random main and interaction}
\end{figure}

\newpage

\section{X follows Chi-square
distribution}\label{x-follows-chi-square-distribution}

In this case, we choose a long-tail distribution. Since the 3rd moment
of Chi-square is no longer 0, so the
\(Cov(X_i^T\beta, X_i^T\Gamma X_i)\) cannot be zero also. The following
simulation is trying to study how this non-zero term affect the
performance of the proposed and original GCTA method.

\subsection{Independent}\label{independent-1}

\subsection{Dependent}\label{dependent}

The procedure of the data generation is following:

\begin{enumerate}
\def\labelenumi{\arabic{enumi}.}
\tightlist
\item
  Generate a vector dependent random variable
  \(Z \sim N(0, \Sigma_{p*})\)
\item
  Square each elements of Z, \(Z^2 = (Z_1^2, \dots, Z_{p*}^2)^T\)
\item
  Divid \(p*\) into \(p\) group and let \(X_i = \sum_{k\in i} Z_{k}^2\),
  where \(i\) all elements in the Ith group.
\item
  Let \(X = (X_1^2, \dots, X_{p}^2)^T\)
\end{enumerate}

\newpage

\subsubsection{\texorpdfstring{\(\beta\) is fixed and \(\Gamma\) is
fixed}{\textbackslash{}beta is fixed and \textbackslash{}Gamma is fixed}}\label{beta-is-fixed-and-gamma-is-fixed-1}

Simulation result

\begin{figure}
\centering
\includegraphics{Simulation_report_files/figure-latex/main_fixed_fixed_chi-1.pdf}
\caption{Main Effect of chi fixed main and interaction}
\end{figure}

\begin{figure}
\centering
\includegraphics{Simulation_report_files/figure-latex/inter_fixed_fixed_chi-1.pdf}
\caption{Interactive Effect of chi fixed main and interaction}
\end{figure}

\clearpage

\subsubsection{\texorpdfstring{\(\beta\) is fixed and \(\Gamma\) is
random}{\textbackslash{}beta is fixed and \textbackslash{}Gamma is random}}\label{beta-is-fixed-and-gamma-is-random-1}

Simulation result

\begin{figure}
\centering
\includegraphics{Simulation_report_files/figure-latex/main_fixed_random_chi-1.pdf}
\caption{Main Effect of chi fixed main and random interaction}
\end{figure}

\begin{figure}
\centering
\includegraphics{Simulation_report_files/figure-latex/inter_fixed_random_chi-1.pdf}
\caption{Interactive Effect of chi fixed main and random interaction}
\end{figure}

\clearpage

\subsubsection{\texorpdfstring{\(\beta\) is random and \(\Gamma\) is
random}{\textbackslash{}beta is random and \textbackslash{}Gamma is random}}\label{beta-is-random-and-gamma-is-random-1}

\begin{figure}
\centering
\includegraphics{Simulation_report_files/figure-latex/main_random_random_chi-1.pdf}
\caption{Main Effect of chi random main and interaction}
\end{figure}

\begin{figure}
\centering
\includegraphics{Simulation_report_files/figure-latex/inter_random_random_chi-1.pdf}
\caption{Interactive Effect of chi random main and interaction}
\end{figure}

\newpage

\subsection{Combine the main and interaction
effect}\label{combine-the-main-and-interaction-effect}

Based on the simulation result of the Chi-square distribution, the
estimation of main and effect are not bad, which means the bias of the
estimations are not large. However, there does exist some bias because
of the non-zero covariance term (\(Cov(X_i^T\beta, X_i^T\Gamma X_i)\)).

One possible solustion or walk-around method is to estimate the total
variance of signal instead of main and interactive variance separately.
Therefore, we don't have to worry about the covariance term between the
main and interative terms.

More specifically, we treat main and interactive covariates as a long
vector and try to estimate its corresponding variance.

\begin{align*}
  Var(Y_i) &= Var(X_i^T\beta + X_i^T \Gamma X_i) + Var(\epsilon_i) \\
           &= Var(X_i^T\beta + X_{i,inter}^T \gamma) + Var(\epsilon_i) \\
           &= \begin{bmatrix}   
                  X_i \\        
                  X_{i,inter}   
              \end{bmatrix}^T     
              \begin{bmatrix}
                \beta \\
                \gamma
              \end{bmatrix} + Var(\epsilon_i) \\
           &= Var(X_{i,total}^T \beta_{total}) + Var(\epsilon_i)
\end{align*}

The big difference of this approch is that the interaction terms are
also standardized and decorrelated.

Do we still have the following equation hold?

\[
  Var(X_{i,total}^T \beta_{total}) =  Var(Z_{i,total}^T \alpha_{total}) 
\]

\newpage

\subsubsection{\texorpdfstring{\(\beta\) is fixed and \(\Gamma\) is
fixed (Chi df =
10)}{\textbackslash{}beta is fixed and \textbackslash{}Gamma is fixed (Chi df = 10)}}\label{beta-is-fixed-and-gamma-is-fixed-chi-df-10}

Simulation result

\begin{figure}
\centering
\includegraphics{Simulation_report_files/figure-latex/main_fixed_fixed_chi_combine_df_10-1.pdf}
\caption{Total effect of chi fixed main and interaction}
\end{figure}

\clearpage

\subsubsection{\texorpdfstring{\(\beta\) is fixed and \(\Gamma\) is
random (Chi df =
10)}{\textbackslash{}beta is fixed and \textbackslash{}Gamma is random (Chi df = 10)}}\label{beta-is-fixed-and-gamma-is-random-chi-df-10}

Simulation result

\begin{figure}
\centering
\includegraphics{Simulation_report_files/figure-latex/main_fixed_random_chi_combine_df_10-1.pdf}
\caption{Total effect of chi fixed main and random interaction}
\end{figure}

\clearpage

\subsubsection{\texorpdfstring{\(\beta\) is random and \(\Gamma\) is
random (Chi df =
10)}{\textbackslash{}beta is random and \textbackslash{}Gamma is random (Chi df = 10)}}\label{beta-is-random-and-gamma-is-random-chi-df-10}

\begin{figure}
\centering
\includegraphics{Simulation_report_files/figure-latex/main_random_random_chi_combine_df_10-1.pdf}
\caption{Total effect of chi random main and interaction}
\end{figure}

\newpage

\subsection{Combine the main and interaction effect with df =
1}\label{combine-the-main-and-interaction-effect-with-df-1}

\subsubsection{\texorpdfstring{\(\beta\) is fixed and \(\Gamma\) is
fixed (Chi df =
1)}{\textbackslash{}beta is fixed and \textbackslash{}Gamma is fixed (Chi df = 1)}}\label{beta-is-fixed-and-gamma-is-fixed-chi-df-1}

Simulation result

\begin{figure}
\centering
\includegraphics{Simulation_report_files/figure-latex/main_fixed_fixed_chi_combine_df_1-1.pdf}
\caption{Total effect of chi fixed main and interaction}
\end{figure}

\clearpage

\subsubsection{\texorpdfstring{\(\beta\) is fixed and \(\Gamma\) is
random (Chi df =
1)}{\textbackslash{}beta is fixed and \textbackslash{}Gamma is random (Chi df = 1)}}\label{beta-is-fixed-and-gamma-is-random-chi-df-1}

Simulation result

\begin{figure}
\centering
\includegraphics{Simulation_report_files/figure-latex/main_fixed_random_chi_combine_df_1-1.pdf}
\caption{Total effect of chi fixed main and random interaction}
\end{figure}

\clearpage

\subsubsection{\texorpdfstring{\(\beta\) is random and \(\Gamma\) is
random (Chi df =
1)}{\textbackslash{}beta is random and \textbackslash{}Gamma is random (Chi df = 1)}}\label{beta-is-random-and-gamma-is-random-chi-df-1}

\begin{figure}
\centering
\includegraphics{Simulation_report_files/figure-latex/main_random_random_chi_combine_df_1-1.pdf}
\caption{Total effect of chi random main and interaction}
\end{figure}

\section{Conclusion}\label{conclusion}

\section{Further work}\label{further-work}


\end{document}
