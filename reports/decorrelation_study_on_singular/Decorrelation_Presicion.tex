\documentclass[]{article}
\usepackage{lmodern}
\usepackage{amssymb,amsmath}
\usepackage{ifxetex,ifluatex}
\usepackage{fixltx2e} % provides \textsubscript
\ifnum 0\ifxetex 1\fi\ifluatex 1\fi=0 % if pdftex
  \usepackage[T1]{fontenc}
  \usepackage[utf8]{inputenc}
\else % if luatex or xelatex
  \ifxetex
    \usepackage{mathspec}
  \else
    \usepackage{fontspec}
  \fi
  \defaultfontfeatures{Ligatures=TeX,Scale=MatchLowercase}
\fi
% use upquote if available, for straight quotes in verbatim environments
\IfFileExists{upquote.sty}{\usepackage{upquote}}{}
% use microtype if available
\IfFileExists{microtype.sty}{%
\usepackage{microtype}
\UseMicrotypeSet[protrusion]{basicmath} % disable protrusion for tt fonts
}{}
\usepackage[margin=1in]{geometry}
\usepackage{hyperref}
\hypersetup{unicode=true,
            pdftitle={Decorrelation methods study},
            pdfauthor={Xuelong Wang},
            pdfborder={0 0 0},
            breaklinks=true}
\urlstyle{same}  % don't use monospace font for urls
\usepackage{graphicx,grffile}
\makeatletter
\def\maxwidth{\ifdim\Gin@nat@width>\linewidth\linewidth\else\Gin@nat@width\fi}
\def\maxheight{\ifdim\Gin@nat@height>\textheight\textheight\else\Gin@nat@height\fi}
\makeatother
% Scale images if necessary, so that they will not overflow the page
% margins by default, and it is still possible to overwrite the defaults
% using explicit options in \includegraphics[width, height, ...]{}
\setkeys{Gin}{width=\maxwidth,height=\maxheight,keepaspectratio}
\IfFileExists{parskip.sty}{%
\usepackage{parskip}
}{% else
\setlength{\parindent}{0pt}
\setlength{\parskip}{6pt plus 2pt minus 1pt}
}
\setlength{\emergencystretch}{3em}  % prevent overfull lines
\providecommand{\tightlist}{%
  \setlength{\itemsep}{0pt}\setlength{\parskip}{0pt}}
\setcounter{secnumdepth}{0}
% Redefines (sub)paragraphs to behave more like sections
\ifx\paragraph\undefined\else
\let\oldparagraph\paragraph
\renewcommand{\paragraph}[1]{\oldparagraph{#1}\mbox{}}
\fi
\ifx\subparagraph\undefined\else
\let\oldsubparagraph\subparagraph
\renewcommand{\subparagraph}[1]{\oldsubparagraph{#1}\mbox{}}
\fi

%%% Use protect on footnotes to avoid problems with footnotes in titles
\let\rmarkdownfootnote\footnote%
\def\footnote{\protect\rmarkdownfootnote}

%%% Change title format to be more compact
\usepackage{titling}

% Create subtitle command for use in maketitle
\providecommand{\subtitle}[1]{
  \posttitle{
    \begin{center}\large#1\end{center}
    }
}

\setlength{\droptitle}{-2em}

  \title{Decorrelation methods study}
    \pretitle{\vspace{\droptitle}\centering\huge}
  \posttitle{\par}
    \author{Xuelong Wang}
    \preauthor{\centering\large\emph}
  \postauthor{\par}
      \predate{\centering\large\emph}
  \postdate{\par}
    \date{2019-07-28}

\usepackage{booktabs}
\usepackage{longtable}
\usepackage{array}
\usepackage{multirow}
\usepackage[table]{xcolor}
\usepackage{wrapfig}
\usepackage{float}
\usepackage{colortbl}
\usepackage{pdflscape}
\usepackage{tabu}
\usepackage{threeparttable}
\usepackage{threeparttablex}
\usepackage[normalem]{ulem}
\usepackage{makecell}

\usepackage{float,amsmath, bbm, siunitx, bm}
\usepackage{pdfpages}
\floatplacement{figure}{H}
\newcommand{\indep}{\rotatebox[origin=c]{90}{$\models$}}

\begin{document}
\maketitle

\section{Motivation}\label{motivation}

To decorrelate the covariate X, so that \(Cov(X) = I\) or
\(Cov(X) = \Sigma_p\), \(\Sigma_p\)'s off diaognal elements are 0s. We
found that after we could decorrelate the data correctly, the GCTA and
EigenPrsim method can work well.

\section{Method}\label{method}

\begin{itemize}
\tightlist
\item
  SVD\\
\item
  PCA\\
\item
  Presicion matrix estimated by glasso and other methods
\end{itemize}

\section{Diffculty}\label{diffculty}

For \(n >p\) case, the SVD method can work well. Basically, we just SVD
to estimate the eigenvalue \(\lambda's\) and find \(1/\sqrt{\lambda}\).
However, when \(n <p\) or even \(n \approx p\), SVD method fails because
the nearly 0 values of \(\lambda's\).

\section{solution}\label{solution}

\subsubsection{Main idea of PCA}\label{main-idea-of-pca}

\paragraph{Singular Value
Decomposition}\label{singular-value-decomposition}

\[
X_s = UDV^T, ~\text{where} ~x_{ij,s} = \frac{x_{ij} - \bar{x}}{s_j}
\]

\begin{itemize}
\tightlist
\item
  \(U = (u_1, \dots, u_r)\) is a n by r orthogonal matrix\\
\item
  \(D = diag(d_1, \dots, d_r)\) is a r by r diagonal matrix\\
\item
  \(V = (v_1, \dots, v_r)\) is a p by r orthogonal matrix
\end{itemize}

\paragraph{Principle Component and
Loading}\label{principle-component-and-loading}

\begin{align*} 
X_s &= \underbrace{\begin{bmatrix} d_1u_1 \hdots  d_ru_r \end{bmatrix} }_\text{PCs}
\underbrace{\begin{bmatrix} v_1^T \\
\vdots \\
v_r^T \\
\end{bmatrix}}_\text{Loading}
\end{align*}

\begin{itemize}
\tightlist
\item
  \(PC_j = d_j\pmb{u_j} = X\pmb{v_j}\) is the jth principle component
\item
  The sample variance of \(PC_j\) is \(d_j^2/n\)
\end{itemize}

\paragraph{Using PCA to reduce the X's
dimension}\label{using-pca-to-reduce-the-xs-dimension}

\[
  X_{s,k} = \sum_{j=1}^{k}d_j\pmb{u_jv_k} = U_kD_kV_k^T,~~ \text{Its Variation} ~~  \sum_{j=1}^kd_j^2/n.
  \]

\[
  X_r(k) = XV_k = UD \begin{bmatrix} 
  V_k^T \\
  V_{(p-k)}^T 
  \end{bmatrix} V_k=  UD \begin{bmatrix} 
  I_k \\
  0 
  \end{bmatrix}= U_kD_k
  \]

\begin{itemize}
\tightlist
\item
  Note that \(Var(X_k) = diag(D_i^2), i \in \{1, \dots, k\}\), which
  means we don't need to decorrelated it again.
\end{itemize}

\paragraph{Simulation set up}\label{simulation-set-up}

\[
X = [x_1^2, \dots, x_p^2]^T
\]

\begin{itemize}
\tightlist
\item
  \(p = 500,800,1000\)\\
\item
  \(x_i \stackrel{iid}{\sim} N(0,1)\)
\end{itemize}

\paragraph{Simulation result}\label{simulation-result}

\begin{itemize}
\tightlist
\item
  If you include all the PCs, the we will get the exact same result
  using original data\\
\item
  The less the number of PCs included, the small the value will be\\
\item
  Need to find a way to choice the number of PC's
\end{itemize}

\subsubsection{Presicion matrix
estimation}\label{presicion-matrix-estimation}

\paragraph{Glasso}\label{glasso}

It seems work for EigenPrism and covariates are normal distribution

\paragraph{dgplasso}\label{dgplasso}

{[} \min\_X \log \det (X) + trace(X \Sigma) +
\rho \textbar{}X\textbar{}\_1

{]}

\paragraph{}\label{section}


\end{document}
